\documentclass[12pt]{article}
\usepackage[utf8]{inputenc}
\usepackage[vietnamese]{babel}
\usepackage{amsmath, amssymb}
\usepackage{listings}
\usepackage{enumitem}
\usepackage{geometry}
\geometry{margin=1in}

\title{Bài Toán 12: Thuật Toán DFS Trên Multigraph (Đồ Thị Đa Cung)}
\author{Đồ Án 5.2: Depth-first Search}
\date{}

\begin{document}

\maketitle

\section*{Phát biểu bài toán}
Cho một đồ thị đa cung $G = (V, E)$, trong đó có thể tồn tại nhiều cạnh giữa cùng một cặp đỉnh $(u, v)$.  
Yêu cầu: Triển khai thuật toán tìm kiếm theo chiều sâu (Depth-First Search – DFS) trên $G$.

\section*{Đặc điểm của multigraph}
\begin{itemize}
    \item Có thể tồn tại nhiều cạnh giữa một cặp đỉnh
    \item Có thể tồn tại cạnh tự khép $(u, u)$
    \item Danh sách kề có thể chứa trùng lặp các đỉnh
\end{itemize}

\section*{Ý tưởng thuật toán}
\begin{itemize}
    \item DFS được triển khai tương tự như với đồ thị đơn
    \item Mỗi đỉnh của một đồ thị được duyệt đúng một lần, điều này vẫn đúng bất kể đồ thị có chứa các cạnh bội (cạnh trùng) hay không.
    \item Dùng mảng \texttt{visited[]} để ngăn việc lặp lại duyệt đỉnh
\end{itemize}

\section*{Thuật toán DFS (pseudocode)}
\begin{verbatim}
DFS_Multigraph(G, u):
    visited[u] ← True
    xử lý đỉnh u
    for mỗi v ∈ adj[u]:  // có thể trùng
        if not visited[v]:
            DFS_Multigraph(G, v)
\end{verbatim}

\section*{Chú thích các biến số}
\begin{itemize}
    \item \texttt{G}: đồ thị đa cung, biểu diễn dưới dạng danh sách kề (có thể chứa trùng)
    \item \texttt{adj[u]}: danh sách các đỉnh kề với $u$ (có thể có lặp)
    \item \texttt{visited[v]}: boolean kiểm tra đỉnh $v$ đã được duyệt chưa
    \item \texttt{u}: đỉnh hiện tại trong DFS
    \item \texttt{res}: danh sách các đỉnh được duyệt theo thứ tự
\end{itemize}

\section*{Xử lý cạnh trùng và self-loop}
\begin{itemize}
    \item Cạnh trùng: nếu $adj[u] = [v, v, v]$ thì DFS vẫn chỉ gọi một lần cho $v$
    \item Cạnh tự khép $(u, u)$: sẽ không gây ra vòng lặp vô hạn nếu kiểm tra \texttt{visited[u]} đúng
\end{itemize}

\end{document}
