\documentclass[12pt]{article}
\usepackage[utf8]{inputenc}
\usepackage[vietnamese]{babel}
\usepackage{amsmath, amssymb}
\usepackage{listings}
\usepackage{enumitem}
\usepackage{geometry}
\geometry{margin=1in}

\title{Bài Toán 11: Thuật Toán DFS Trên Đồ Thị Đơn Hữu Hạn}
\author{Đồ Án 5.2: Depth-first Search}
\date{}

\begin{document}

\maketitle

\section*{Phát biểu bài toán}
Cho đồ thị đơn hữu hạn $G = (V, E)$ (finite simple graph).  
Yêu cầu: Triển khai thuật toán duyệt theo chiều sâu (Depth-First Search – DFS) bắt đầu từ một đỉnh $s$.

\section*{Ý tưởng}
\begin{itemize}
    \item DFS đi càng sâu càng tốt trước khi quay lại duyệt các đỉnh còn lại
    \item Sử dụng đệ quy hoặc stack để triển khai
    \item Trên đồ thị đơn, mỗi cạnh tồn tại một lần duy nhất → DFS sẽ duyệt tối đa $O(V + E)$
\end{itemize}

\section*{Thuật toán DFS (pseudocode)}
\begin{verbatim}
DFS(G, u):
    visited[u] ← True
    xử lý đỉnh u
    for mỗi đỉnh v ∈ adj[u]:
        if not visited[v]:
            DFS(G, v)
\end{verbatim}

\section*{Chú thích các biến số}
\begin{itemize}
    \item \texttt{G}: đồ thị đầu vào, là đồ thị đơn
    \item \texttt{adj[u]}: danh sách kề của đỉnh $u$
    \item \texttt{visited[u]}: đánh dấu đỉnh đã được duyệt
    \item \texttt{u}: đỉnh bắt đầu DFS
    \item \texttt{res}: danh sách kết quả thứ tự duyệt
\end{itemize}

\section*{Đặc điểm đồ thị đơn}
\begin{itemize}
    \item Mỗi cặp đỉnh chỉ có nhiều nhất một cạnh
    \item Không có cạnh tự khép $(u,u)$
    \item Đồ thị vô hướng ( auto mặc định)
\end{itemize}

\end{document}
