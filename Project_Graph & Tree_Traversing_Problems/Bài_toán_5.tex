\documentclass[12pt]{article}
\usepackage[utf8]{inputenc}
\usepackage[vietnamese]{babel}
\usepackage{amsmath, amssymb}
\usepackage{listings}
\usepackage{enumitem}
\usepackage{geometry}
\geometry{margin=1in}

\title{Bài Toán 5: Giải Các Bài Tập Đồ Thị (Val21)}
\author{Đồ Án 4: Duyệt Đồ Thị \& Cây}
\date{}

\begin{document}

\maketitle

\section*{Phát biểu bài toán}
Giải các bài toán cơ bản từ mục 1.1 đến 1.6 và bài tập 1.1 đến 1.10 theo tài liệu \textbf{Val21}.  
Chúng bao gồm các thao tác với đồ thị đơn như:
\begin{itemize}
    \item Tính bậc của các đỉnh
    \item Kiểm tra đồ thị vô hướng / có hướng
    \item Kiểm tra liên thông
    \item Kiểm tra đồ thị có phải cây hay không
    \item Đếm số thành phần liên thông
    \item Duyệt đồ thị bằng DFS / BFS
\end{itemize}

\section*{Ý tưởng thuật toán}
\begin{enumerate}
    \item \textbf{Bậc đỉnh:} Với đồ thị vô hướng, bậc đỉnh là số lượng đỉnh kề. Với đồ thị có hướng, dùng \texttt{in-degree} và \texttt{out-degree}.
    \item \textbf{Liên thông:} Dùng DFS để kiểm tra xem có thể duyệt hết tất cả các đỉnh từ một đỉnh gốc.
    \item \textbf{Cây:} Đồ thị là cây nếu:
        \begin{itemize}
            \item Liên thông
            \item Không có chu trình
            \item Có đúng \( n-1 \) cạnh với \( n \) đỉnh
        \end{itemize}
    \item \textbf{DFS/BFS:} Tiêu chuẩn để duyệt toàn bộ đồ thị, áp dụng để kiểm tra tính liên thông hoặc in thứ tự duyệt.
\end{enumerate}

\section*{Chú thích các biến số}
\begin{itemize}
    \item \texttt{n}: số lượng đỉnh trong đồ thị
    \item \texttt{adj}: danh sách kề, kiểu \texttt{adj[i]} là danh sách các đỉnh kề với đỉnh $i$
    \item \texttt{visited[i]}: mảng đánh dấu đỉnh $i$ đã được duyệt trong DFS/BFS
    \item \texttt{inDeg[i]}: số lượng cung đi vào đỉnh $i$ (đồ thị có hướng)
    \item \texttt{outDeg[i]}: số lượng cung đi ra từ đỉnh $i$
    \item \texttt{parent[i]}: đỉnh cha của $i$ trong DFS tree
    \item \texttt{component\_count}: số thành phần liên thông
    \item \texttt{isTree}: cờ kiểm tra đồ thị có phải là cây hay không
\end{itemize}

\end{document}
