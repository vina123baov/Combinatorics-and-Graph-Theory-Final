\documentclass[12pt]{article}
\usepackage[utf8]{inputenc}
\usepackage[vietnamese]{babel}
\usepackage{amsmath, amssymb}
\usepackage{listings}
\usepackage{enumitem}
\usepackage{geometry}
\geometry{margin=1in}

\title{Bài Toán 4: Chuyển Đổi Giữa Các Biểu Diễn Đồ Thị và Cây}
\author{Đồ Án 4: Duyệt Đồ Thị & Cây}
\date{}

\begin{document}

\maketitle

\section*{Phát biểu bài toán}
Viết chương trình C/C++, Python để chuyển đổi giữa:
\begin{itemize}
    \item 4 dạng biểu diễn đồ thị: 
    \begin{enumerate}[label=(\alph*)]
        \item adjacency matrix
        \item adjacency list
        \item extended adjacency list (với trọng số)
        \item adjacency map
    \end{enumerate}
    \item 3 dạng biểu diễn cây:
    \begin{enumerate}[label=(\alph*)]
        \item array of parents
        \item first-child next-sibling
        \item graph-based tree
    \end{enumerate}
\end{itemize}

Tổng cộng có $\binom{4}{2} \cdot 2 + \binom{3}{2} \cdot 2 = 42$ chuyển đổi.

\section*{Phân tích tổng quát}
\begin{itemize}
    \item \textbf{Đồ thị:} Dùng cấu trúc lớp `Graph` để chứa dữ liệu dưới các dạng khác nhau.
    \item \textbf{Cây:} Sử dụng cây gốc không có chu trình, áp dụng các phương pháp duyệt và ánh xạ.
\end{itemize}

\section*{Thuật toán}
Mỗi chuyển đổi cần:
\begin{enumerate}
    \item Hàm đọc từ định dạng A
    \item Hàm sinh định dạng B
\end{enumerate}

Ví dụ: để chuyển từ adjacency matrix → adjacency list:
\begin{enumerate}
    \item Duyệt từng dòng i
    \item Với mỗi cột j nếu matrix[i][j] == 1 thì thêm j vào list[i]
\end{enumerate}

\section*{Chú thích các biến số}
\begin{itemize}
    \item \texttt{matrix[i][j]}: lưu 1 nếu có cạnh từ i đến j
    \item \texttt{adjList[i]}: vector danh sách kề
    \item \texttt{parent[i]}: cây biểu diễn bằng mảng cha
\end{itemize}

\section*{Ví dụ minh họa}
Cho đồ thị 4 đỉnh:
\[
A = 
\begin{bmatrix}
0 & 1 & 0 & 1\\
1 & 0 & 1 & 0\\
0 & 1 & 0 & 1\\
1 & 0 & 1 & 0\\
\end{bmatrix}
\Rightarrow 
\text{adjList = } 
\{
0: [1,3], 1:[0,2], 2:[1,3], 3:[0,2]
\}
\]

\end{document}
