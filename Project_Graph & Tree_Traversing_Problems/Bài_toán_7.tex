\documentclass[12pt]{article}
\usepackage[utf8]{inputenc}
\usepackage[vietnamese]{babel}
\usepackage{amsmath, amssymb}
\usepackage{listings}
\usepackage{enumitem}
\usepackage{geometry}
\geometry{margin=1in}

\title{Bài Toán 7: Tree Traversal – Duyệt Cây}
\author{Đồ Án 4: Duyệt Đồ Thị \& Cây}
\date{}

\begin{document}

\maketitle

\section*{Phát biểu bài toán}
Cho một cây gốc $T$, viết chương trình C/C++, Python để duyệt cây theo 4 cách:
\begin{enumerate}[label=(\alph*)]
    \item \textbf{Preorder traversal (tiền thứ tự)}: Duyệt node → con trái → con phải
    \item \textbf{Postorder traversal (hậu thứ tự)}: Duyệt con trái → con phải → node
    \item \textbf{Top-down traversal}: Duyệt theo tầng từ gốc xuống lá (BFS)
    \item \textbf{Bottom-up traversal}: Duyệt theo tầng từ lá lên gốc
\end{enumerate}

\section*{Ý tưởng}
\begin{itemize}
    \item Preorder và Postorder: dùng đệ quy truyền thống
    \item Top-down: dùng queue để duyệt theo tầng (BFS)
    \item Bottom-up: dùng BFS để lưu tầng, sau đó in ngược lại
\end{itemize}

\section*{Chú thích các biến số}
\begin{itemize}
    \item \texttt{Node}: class chứa \texttt{label} và \texttt{children}
    \item \texttt{tree}: cây gốc cần duyệt
    \item \texttt{result}: mảng kết quả duyệt
    \item \texttt{queue}: dùng trong top-down duyệt theo tầng
    \item \texttt{levels}: danh sách lưu từng tầng
\end{itemize}


\end{document}
