\documentclass[12pt]{article}
\usepackage[utf8]{inputenc}
\usepackage[vietnamese]{babel}
\usepackage{amsmath, amssymb}
\usepackage{listings}
\usepackage{enumitem}
\usepackage{geometry}
\geometry{margin=1in}

\title{Bài Toán 9: Thuật Toán BFS Trên Multigraph (Đồ Thị Đa Cung)}
\author{Đồ Án 5.1: Breadth-first Search}
\date{}

\begin{document}

\maketitle

\section*{Phát biểu bài toán}
Cho một đồ thị đa cung hữu hạn $G = (V, E)$, trong đó có thể tồn tại nhiều cạnh nối giữa cùng một cặp đỉnh $(u,v)$.  
Yêu cầu: Triển khai thuật toán tìm kiếm theo chiều rộng (Breadth-First Search – BFS) trên $G$.

\section*{Đặc điểm của multigraph}
\begin{itemize}
    \item Có thể có nhiều hơn một cạnh giữa 2 đỉnh $u$ và $v$
    \item Có thể có cạnh tự khép (self-loop), ví dụ $(u,u)$
    \item Danh sách kề có thể chứa nhiều lần cùng một đỉnh kề
\end{itemize}

\section*{Ý tưởng}
\begin{itemize}
    \item Duyệt theo chiều rộng như đồ thị đơn
    \item Để tránh duyệt lặp qua nhiều cạnh trùng nhau, cần dùng mảng \texttt{visited[]} để đánh dấu đã thăm
    \item Mỗi đỉnh chỉ được duyệt đúng một lần, bỏ qua các cạnh trùng nếu đỉnh kề đã thăm
\end{itemize}

\section*{Thuật toán BFS (pseudocode)}
\begin{verbatim}
BFS_Multigraph(G, s):
    visited[v] ← False với mọi v ∈ V
    Q ← hàng đợi rỗng
    visited[s] ← True
    enqueue(Q, s)
    while Q không rỗng:
        u ← dequeue(Q)
        xử lý đỉnh u
        for mỗi v ∈ adj[u]:  // cho phép lặp
            if not visited[v]:
                visited[v] ← True
                enqueue(Q, v)
\end{verbatim}

\section*{Chú thích các biến số}
\begin{itemize}
    \item \texttt{G}: đồ thị đa cung, biểu diễn bằng danh sách kề có thể chứa trùng
    \item \texttt{adj[u]}: danh sách các đỉnh kề với $u$, có thể chứa $v$ nhiều lần
    \item \texttt{s}: đỉnh bắt đầu BFS
    \item \texttt{visited[v]}: boolean đánh dấu đỉnh $v$ đã được duyệt
    \item \texttt{Q}: hàng đợi FIFO dùng để duyệt BFS
    \item \texttt{res}: danh sách kết quả duyệt
\end{itemize}

\section*{Giải thích xử lý trùng cạnh}
\begin{itemize}
    \item Tại vì một đỉnh $v$ có thể xuất hiện nhiều lần trong \texttt{adj[u]}, ta chỉ xét \texttt{visited[v]} đúng một lần
    \item Ví dụ: nếu $adj[0] = [1,1,2]$, thì đỉnh 1 chỉ được duyệt 1 lần duy nhất
    \item Điều này sẽ giúp BFS vẫn có độ phức tạp $O(V + E)$ với $E$ là tổng số cung (bao gồm trùng cạnh)
\end{itemize}

\end{document}
