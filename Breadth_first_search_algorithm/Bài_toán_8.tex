\documentclass[12pt]{article}
\usepackage[utf8]{inputenc}
\usepackage[vietnamese]{babel}
\usepackage{amsmath, amssymb}
\usepackage{listings}
\usepackage{enumitem}
\usepackage{geometry}
\geometry{margin=1in}

\title{Bài Toán 8: Thuật Toán BFS Trên Đồ Thị Đơn Hữu Hạn}
\author{Đồ Án 5.1: Breadth-first Search}
\date{}

\begin{document}

\maketitle

\section*{Phát biểu bài toán}
Cho một đồ thị đơn hữu hạn $G = (V, E)$ (finite simple graph). Viết chương trình C/C++, Python để thực hiện thuật toán tìm kiếm theo chiều rộng (Breadth-First Search - BFS) trên $G$.

\section*{Ý tưởng}
\begin{itemize}
    \item Duyệt đồ thị bắt đầu từ đỉnh nguồn $s$
    \item Dùng hàng đợi (queue) để duyệt từng đỉnh theo thứ tự vào-trước-ra-trước
    \item Đánh dấu các đỉnh đã thăm để tránh lặp
\end{itemize}

\section*{Thuật toán BFS (pseudocode)}
\begin{verbatim}
BFS(G, s):
    Tạo hàng đợi Q
    visited[s] ← true
    Q.enqueue(s)
    while Q không rỗng:
        u ← Q.dequeue()
        xử lý đỉnh u
        for mỗi đỉnh v kề với u:
            if not visited[v]:
                visited[v] ← true
                Q.enqueue(v)
\end{verbatim}

\section*{Chú thích các biến số}
\begin{itemize}
    \item \texttt{G}: đồ thị đầu vào, dưới dạng danh sách kề (adjacency list)
    \item \texttt{s}: đỉnh bắt đầu BFS
    \item \texttt{visited[i]}: mảng boolean đánh dấu đỉnh $i$ đã được thăm
    \item \texttt{queue}: hàng đợi FIFO lưu các đỉnh đang chờ duyệt
    \item \texttt{res}: danh sách thứ tự các đỉnh được duyệt
\end{itemize}

\end{document}
